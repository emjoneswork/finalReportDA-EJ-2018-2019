%======================================================================
\chapter{Computer Simulations using V-REP}
\label{chap:VREP}
%======================================================================
